\begin{abstractzh}
本論文提出一方法整合了全球衛星定位系統 (GPS)、
姿態航向參考系統 (AHRS)
與雷射掃描式測距儀(LiDAR),
以達成無人自走車自動導航與避開導航路徑上之障礙物的功能。
利用GPS和AHRS,
車輛可得知本身與目標點之間的相對關係和方向資訊,
達成自動導航的功能。
使用雷射掃描式測距儀,
車輛便可在導航的過程中量測週遭環境之變化,
以便規劃其它路徑來閃避障礙物。
本論文開發的Yun-Trooper II使用四驅模型搖控車作為底盤,
並使用ARM處理器架構之Linux嵌入式系統作為決策與控制核心,
而軟體方面則使用C++進行運算和控制車輛運動。

\keywordszh{GPS、AHRS、Linux、雷射測距儀、無人自走車}

\end{abstractzh}

\begin{abstracten}
This thesis presents a method for the navigation and obstacle avoidance 
of an Unmanned Ground Vehicle (UGV), which integrated the
Global Positioning System (GPS), 
Attitude and Heading Refrence System (AHRS) and
LiDAR rangefinder.
From the data collected by GPS and AHRS sensor, position of the target relative
to vehicle itself and heading of the vehicle could be determined,
which is the information exploited by the navigation algorithm.
With a LiDAR rangefinder, the vehicle could perceive the environment, therefore
the algorithm could planning a new path if some obstacles have blocked the original path.
This thesis has developed an UGV named `Yun-Trooper II', which is based on a 4-wheel-drived
remote-controlled model car. It has an ARM-Based embedded Linux computer as
the core of calculation and control, and all the program were developed by C++ programming
language.

\keywordsen{GPS, AHRS, LiDAR, Linux, UGV}

\end{abstracten}